\documentclass{article}

\title{The Architecture of the Designer}
\author{John B. Zuckerman}

\usepackage{ifthen}

%
% Adapted from macros for R^nRS by John B. Zuckerman
%

\makeatletter

% Chapters, sections, etc.

\newcommand{\extrapart}[1]{
  \chapter*{#1}
  \markboth{#1}{#1}
  \vskip 1ex
  \addcontentsline{toc}{chapter}{#1}}

\newcommand{\clearchapterstar}[1]{
  \clearpage
  \topnewpage[
    \centerline{\large\bf\uppercase{#1}}
    \bigskip]}

\newcommand{\clearextrapart}[1]{
  \clearchapterstar{#1}
  \markboth{#1}{#1}
  \addcontentsline{toc}{chapter}{#1}}



\newcommand{\vest}{}
\newcommand{\dotsfoo}{$\ldots\,$}

\newcommand{\sharpfoo}[1]{{\tt\##1}}
\newcommand{\schfalse}{\sharpfoo{f}}
\newcommand{\schtrue}{\sharpfoo{t}}

\newcommand{\singlequote}{{\tt'}}  %\char19
\newcommand{\doublequote}{{\tt"}}
\newcommand{\backquote}{{\tt\char18}}
\newcommand{\backwhack}{{\tt\char`\\}}
\newcommand{\atsign}{{\tt\char`\@}}
\newcommand{\sharpsign}{{\tt\#}}
\newcommand{\verticalbar}{{\tt|}}

\newcommand{\coerce}{\discretionary{->}{}{->}}
\newcommand{\okbrk}{\discretionary{}{}{}}   % ok to break without inserting hyphen

\def\elem{\hbox{\raise.13ex\hbox{$\scriptstyle\in$}}}

\newcommand{\meta}[1]{{\noindent\hbox{\rm$\langle$#1$\rangle$}}}
\let\hyper=\meta

\newcommand{\lambdaexp}{lambda expression}
\newcommand{\Lambdaexp}{Lambda expression}
\newcommand{\cname}[1]{{\sf $<$#1$>$}}

%------------------------------------------------------------------
%
% mysterious stuff from the r4rs.
%
% \frobq will make quote and backquote look nicer.
\def\frobqcats{%\catcode`\'=13
\catcode`\`=13{}}
{\frobqcats
\gdef\frobqdefs{%\def'{\singlequote}
\def`{\backquote}}}
\def\frobq{\frobqcats\frobqdefs}

% \cf = code font
% Unfortunately, \cf \cf won't work at all, so don't even attempt to
% next constructions which use them...
\newcommand{\cf}{\frenchspacing\frobq\tt}

% Same as \obeycr, but doesn't do a \@gobblecr.
{\catcode`\^^M=13 \gdef\myobeycr{\catcode`\^^M=13 \def^^M{\\}}%
\gdef\restorecr{\catcode`\^^M=5 }}

{\catcode`\^^I=13 \gdef\obeytabs{\catcode`\^^I=13 \def^^I{\hbox{\hskip 4em}}}}

{\obeyspaces\gdef {\hbox{\hskip0.5em}}}

\gdef\gobblecr{\@gobblecr}

\def\setupcode{\@makeother\^}

%------------------------------------------------------------------

% Scheme example environment

\newenvironment{schemenoindent}{
  % Commands for scheme examples
  \newcommand{\ev}{\>\>\>\evalsto}       % three tabs
  \newcommand{\wev}{\>\>\>\>\evalsto}    % four tabs
  \newcommand{\wwev}{\>\>\>\>\>\evalsto} % five tabs
  \newcommand{\lev}{\\\>\evalsto}        % one tab
  \newcommand{\unspecified}{{\em{}unspecified}}
  \newcommand{\scherror}{{\em{}error}}
  \setupcode
  \small \cf \obeytabs \obeyspaces \myobeycr
  \begin{tabbing}%
\qquad \= \hspace*{4.5em} \= \hspace*{4.5em} \= \hspace*{4.5em} \= \hspace*{4.5em} \= \hspace*{4.5em} \= \kill
\gobblecr}{\end{tabbing}\unskip}


\newenvironment{scheme}{
  % Commands for scheme examples
  \newcommand{\ev}{\>\>\>\evalsto}       % three tabs
  \newcommand{\wev}{\>\>\>\>\evalsto}    % four tabs
  \newcommand{\wwev}{\>\>\>\>\>\evalsto} % five tabs
  \newcommand{\lev}{\\\>\evalsto}        % one tab
  \newcommand{\unspecified}{{\em{}unspecified}}
  \newcommand{\scherror}{{\em{}error}}
  \setupcode
  \small \cf \obeyspaces \myobeycr
  \begin{tabbing}%
\qquad \= \hspace*{4.5em} \= \hspace*{4.5em} \= \hspace*{4.5em} \= \hspace*{4.5em} \= \hspace*{4.5em} \= \+ \kill
\gobblecr}{\end{tabbing}\unskip}

\newcommand{\evalsto}{$\Longrightarrow$}

%
% The Proc Environment - arguments are:
%
% 1 - proc name 
% 2 - formals list
% 3 - return formal
%
\newenvironment{proc}[3]
   {\bigskip\goodbreak
   \noindent{\sf (#1{\sl #2})} \hfill {Scheme Procedure}\par{\indent
   Returns: \ifthenelse{\equal{#3}{}}{\em unspecified}{\em #3}
   }\par
   \begin{list}{}{
   \setlength{\leftmargin}{3ex}
   \setlength{\listparindent}{0pt}
   }\item[] 
   \penalty 10000
   }{
   \end{list}
   }


%
% The Method Environment - arguments are:
%
% 1 - proc name 
% 2 - formals list
% 3 - return formal
%
\newenvironment{method}[3]
   {\medskip\goodbreak
   \noindent{\sf (#1{\sl #2})} \hfill {Method}\par{\indent
   Returns:    \ifthenelse{\equal{#3}{}}{\em unspecified}{\em #3}
   }\par
   \begin{list}{}{
   \setlength{\leftmargin}{3ex}
   \setlength{\listparindent}{0pt}
   }\item[] 
   \penalty 10000
   }{
   \end{list}
   }


%
% The Class Environment - arguments are:
%
% 1 - class name
% 2 - parent name
% 3 - class type
%
\newenvironment{class}[3]
   {\bigskip\bigskip\bigskip\bigskip\goodbreak
    \noindent{\sf \cname{#1}}
    \ifthenelse{\equal{#2}{}}{}{ (parent: \cname{#2})} 
     \hfill {#3 Class}
   \par
   \begin{list}{}{
   \setlength{\leftmargin}{3ex}
   \setlength{\listparindent}{0pt}
   }\item[] 
   \penalty 10000
   }{
   \end{list}
   }





\begin{document}

\maketitle


\section{Introduction}

The designer is a single program written for the {\sc SWL}\cite{SWL}
programming environment (Chez Scheme\cite{TSPL} combined with the
graphics user interface capabilities of Tcl/Tk\cite{TCLTK}).  The
designer permits a user to code and edit a hardware design expressed
in a purely functional subset of Scheme, and to manipulate it using
various meaning- (or correctness-) preserving transformations provided
by the designer.  This report documents the architecture of the
designer to provide the information needed for future development
work.

The designer's user interface operates in either of two modes: Unlocked
and Locked.  In Unlocked Mode, the designer functions much like a
standard text editor, with some extensions to support parenthesis
balancing and structure collapsing (outlining).  In Locked Mode, the
meaning-preserving transformations are enabled and unrestricted text
editing is disabled.

In Locked Mode the designer records the current state of the design
prior to each user-selected meaning-preserving transformation.  The
designer displays a representation of all prior design states in a
separate window (``Design History'') in the form of a tree diagram.
Each design version is displayed as a tree node annotated with the
type and values employed in the transformation.  The user may visit
previous design states simply by clicking on the appropriate node in
the Design History window.

The implementation of the designer itself heavily exploits the object
oriented language extension to Scheme provided in {\sc SWL}, which is a
relatively conventional class-based single-inheritance OOP system.  A
{\sc SWL} class encapsulates a set of methods and instance variables (local
state), and provides a mechanism for creating new instances of the
class (each with its own set of instance variables).


\section{Selected Features of the Designer}

The designer is a specialized version of another tool: the {\sc SWL}
general purpose Scheme language editing tool (the {\em editor}), which
was developed earlier for this project.

The editor provides a conventional GUI application layout, with a
scrollable and resizeable multi-line window for editing text, a
one-line ``mini-buffer'' for displaying messages, and a menu-bar with
pull-down menu items and buttons.  The editor also displays the name
of the file being edited, and a ``worried face'' icon if editing
changes have been made since the contents of the text buffer were last
saved to disk.

\subsection{Locked and Unlocked Modes}

The designer always operates in one of two mutually exclusive modes:
{\em Unlocked} and {\em Locked}.  Unlocked Mode provides the editing
capabilities of an ordinary text editor with none of the
meaning-preserving tranformations or editing restrictions of Locked Mode.

\subsubsection{Unlocked Mode}

The editor provides support for Scheme expression parenthesis
balancing and lexical structure outlining (a feature known as ``the
box'').  In Unlocked Mode the designer adds support for structure
collapsing and expanding, also known as {\em outlining}.

The user input Right-Button (i.e., pressing and releasing the
rightmost mouse button) is used both to collapse and expand a
parenthesized form.  Shift-Right-Button (i.e., pressing and releasing
the rightmost mouse button plus either shift key) is used to expand
all collapsed forms (outer and nested) at the current cursor position.


\subsubsection{Locked Mode}

Once a complete design has been entered into the designer, the user
may enter Locked Mode by pressing the ``Lock'' button on the menu-bar.
When Locked Mode is selected, the designer parses the design code,
and:

\begin{itemize}
\item if the design is not syntactically well formed, reports an error
by (1) highlighting in red the first character position where the
parser detected an error, and (2) by displaying the parser's error
message in the editor's mini-buffer,

\item if the design is syntactically well formed, highlights certain
recognized subforms of the design in various colors (see table
\ref{tab:highlight}).

\end{itemize}

Once the user's design has been accepted by the parser, the user may
apply various meaning-preserving design transformations to the
design-in-progress, and may revert that design to an earlier version
by using the ``Design History'' tool.

\subsection{Design Transformations}

At present, two meaning-preserving design transformations are
supported by the designer. A larger number of transformations are
envisioned for the future.  The supported transformations are:

\begin{description}
\item[substitution] The bound expression of a lexical variable may be
substituted for any of its variable references simply by clicking on a
reference.  A popup menu is presented, which lists applicable
operations.  Selecting the ``Substitute'' popup menu option performs
the substitution.

\item[binding relocation] A lexical binding may be relocated within
its {\sf let} form, or to another {\sf let} form altogether (thereby
changing its scope).  First, the user moves the mouse cursor over a
binding to select it. A selected binding is highlighted in pink. Next,
the user clicks the mouse left-button to cause the highlighted region
to begin blinking.  Finally, the user clicks the mouse left-button
within the scope of another binding of the current {\sf let} form, or
within another {\sf let} form, to cause the selected binding to be
relocated.

At present, the designer does not implement knowledge of the full
semantics of bindings for all {\sf let} form variants. Therefore, at
present the designer permits certain incorrect relocations.
Additional constraints on binding relocation for correctness may be
added to the designer in the future.

\end{description}



\subsection{The Design History Tool}

The designer displays the complete history of Locked Mode design
versions as a {\em tree diagram\/} in a separate window (labeled
``Design History'').  Each design version is represented as a node of
the tree, and is annotated with information about the transformation
used to derive the version.  The user may visit previous design states
simply by clicking with the mouse on the appropriate node in the
diagram.

When the user selects Locked Mode, the edit-buffer contents are
parsed.  If parsing is successful, a green-colored node is added to
the tree diagram.  Otherwise, a red node is added.  The user may then
select any of the available design transformations; a node is added to
the tree diagram for each transformation selected.  A transformation
node is colored black and is annotated to the right with the type and
information about the subexpression used in the selected
transformation.  A node that is drawn {\em below\/} another node
denotes a successor version, or {\em child}, of the node above
immediately it.

A square box is drawn around one tree node to denote the design
version that is currently selected and resident in the edit-buffer.
The user may select other versions at any time by clicking on a tree
node with the left mouse button.  The box is relocated to denote the
newly selected version.

If the user first selects an interior node of the tree, and then
selects a design transformation, a new {\em branch\/} of the tree is
created and added to the diagram.  Branches are always drawn so that
there is there is space available to the right of the node for the
display of annotation information.  Hence, sibling nodes do not always
appear at the same vertical coordinate in a tree diagram. Tree
branching is denoted by drawing an open circle as a successor node,
with branches extending down and to the right.



\begin{table}\centering
\begin{tabular}[]{|l|l|}
\hline
yellow & lexical variable reference     \\
brown & bound lexical variable   \\
pink  & lexical binding     \\
orange & matched parenthesis \\
red  & parsing error     \\
\hline
\end{tabular}
\caption{Highlighting Colors used in Locked Mode}\label{tab:highlight}
\end{table}




\section{A Brief Overview of the {\sc SWL} OOP System}

The {\sc SWL} OOP system is implemented as a set of syntactic extensions to
Chez Scheme; it is not integrated with the Scheme type system, e.g.,
Scheme objects such as pairs and integers are not instances of an OOP
system class and are therefore unaffected by the class system.
Furthermore, {\sc SWL} classes and instances are themselves represented as
native Scheme objects (vectors).  Thus, the {\sc SWL} OOP system extends
Scheme in the same heterogeneous way that the C++ class mechanism
extends the C language.

A new {\sc SWL} class is defined by using the {\sf define-class} syntactic form
provided by {\sc SWL}:

\begin{scheme}
(define-class (<class-name> ...)  \>\>\>\>\>	SWL Syntactic Form
  (<parent-name> ...)	
  (ivars ...)
  (inherited ...)
  (inheritable ...)
  (private ...)
  (protected ...)
  (public ...))
\end{scheme}

\cname{class-name} is the name of the new class (a Scheme symbol) and
\cname{parent-name} is the name of the parent class, if any.  These names
may be followed by zero or more instance initialization arguments.

{\sf ivars\/} is a list of instance variables.  Each instance of the
class will possess a private set of these variables.  {\sf inherited\/} is 
the set of of instance variables that are to be
inherited from the parent class.  {\sf inheritable} is the subset of
the ivars that may be inherited by a subclass, if any.  {\sf private},
{\sf protected}, and {\sf public} sections contain instance methods
(i.e., Scheme procedures that may access instance the local state
variables).

Private and protected methods are callable as ordinary Scheme
procedures, but only from within the lexical scope of a class
definition (private methods only within the scope of the current
class, protected methods only within the scope of the current class or
any of its subclasses).  Public methods are callable from outside the
class using the message passing form {\sf send,} which takes an
instance of the class as its first argument.

The {\sc SWL} syntactic form {\sf create} is used to instantiate a
class.  Additional arguments to {\sf create} are passed as instance
initialization arguments to the class.  A convention for passing
keyword arguments (the {\sf with} clause) is provided.

See the {\sc SWL} documentation\cite{SWL} for more details.




\section{An Overview of the Designer Architecture}


The designer is implemented by specializing (i.e., subclassing) an
existing tool, the editor.  The designer adds knowledge of a
functional subset of the Scheme language, which is used as a hardware
description language.  The designer uses the knowledge of the
functional language to annotate syntactic elements of a user's design,
and to provide the user with meaning-preserving
design transformations.

As {\sc SWL} is a multi-threaded program execution environment, it
supports the simultaneous execution of multiple applications.  This
may include multiple instances of the designer application, and we
will refer to separately executing designers as ``designer
instances.''

\subsection{The Procedure {\sf new-design}}

The procedure {\sf new-design} creates a new executing instance of the
designer application.  {\sf new-design} may be called with zero or one
argument.  If no argument is supplied, a new designer instance is
started with an empty edit-buffer.  If supplied, the argument must be
the path of a file containing a design; the design is then loaded into
the edit-buffer.  {\sf new-design} creates all of the GUI components
of the new instance; most of its complexity is associated with
creating the menu-bar items.

{\sf new-design} invokes the {\sc SWL} procedure {\sf
swl:make-application}, which is used to create a separate
event-processing thread, or ``fallback queue,'' for the new designer
instance.  Creating a separate event-processing thread for a new
designer instance is optional, however, it has the advantage of
separating the scheduling of the instance's event processing from that
of other designer instances or other unrelated {\sc SWL} applications.
{\sf new-design} also creates a new transient {\sc SWL} thread which
builds the set of character fonts available on the ``Preferences''
menu-bar item.

Two key menu-bar buttons are labeled Lock and Unlock.  Pressing Lock
disables general purpose text editing and causes the designer to
interpret the edit-buffer contents as a design description (or
``design'' for short).  Pressing Unlock causes the designer to revert
to a general-purpose text editor.

\subsection{The Procedure {\sf parse-it}}

To interpret a design the designer begins by parsing the edit-buffer
contents into grammatical elements. When the user presses Lock, the
procedure {\sf parse-it} reads the contents of the designer's
edit-buffer and parses it according to the grammar described below in
table \ref{tab:grammar}.  As a side-effect, the parser creates an
annotated {\em parse tree}, which contains lexical information (row
and column values), terminal and non-terminal data from the grammar,
etc.  Also as a side-effect, the parser also annotates the user's
program by highlighting certain syntactic elements in color, or by
displaying information about a syntactic element in the editor's
mini-buffer.

{\sf parse-it} could be extended (or an additional design analysis
pass provided) to perform additional syntactic verification. For
example, {\sf parse-it} could verify that no two bindings of a {\sf let}
form, or no two formals of a procedure declaration, share the same
variable name.

The most important data structures created during the annotation
process performed by {\sf parse-it} are instances of the class
\cname{design-markup}.  Full details for \cname{design-markup} are
given in a separate section below.  Each \cname{design-markup}
instance represents a single node of the parse tree.  It encapsulates
a wide variety of information, including the corresponding raw source
expression, parent and subordinate parse trees (i.e.,
\cname{design-markup} instances), row and column position within the
text buffer, and the syntactic category (i.e.,
the grammar nonterminal) of the parse tree node.




\section{The Designer OOP Class Structure}

The designer makes extensive use of existing {\sc SWL} classes, and also
defines several new classes of its own.  A general discussion of the
{\sc SWL} class system is available in \cite{SWL}; here we will focus on the
{\sc SWL} classes and methods most relevant to the designer, and on the
newly defined classes that are unique to the designer.

At the coarsest architectural level, the designer may be viewed as a
specialization of a more general tool: the editor.  Both the editor
and the designer introduce the new OOP classes described below.

Each described class is partitioned into one of four categories:

\begin{description}

\item[Core]	is a feature that is intrinsic to {\sc SWL} proper.

\item[Support] is a feature that is not intrinsic to {\sc SWL}, but is found in
	  several other {\sc SWL} applications.  (These features typically
	  are defined in files located in the ``common'' subdirectory of a {\sc SWL} 
	  release directory).

\item[Editor]  is a feature that was developed for the editor.

\item[Designer] is a feature that was developed specifically for the
	  designer.

\end{description}




\begin{class}{text}{}{Core}

\cname{text} is the parent of all more-specialized text buffer widget
classes.  The designer uses two instances of a \cname{text} subclass, one
for the main edit window that contains the user's design code, and one
for the {\em mini-buffer}, i.e., the one-line message window immediately
above the edit window.


\paragraph{Offsets vs. Indexes}

A location within a text widget buffer is often represented in either
of two ways: as an offset (in characters) from the beginning of the
text buffer (with each ``newline'' counting as one character), or as a
Scheme pair whose {\sf car} is the column (x coordinate) and whose
{\sf cdr} is the row (y coordinate).

Most {\sc SWL} methods expect location arguments to be represented as
indexes (pairs), but sometimes it is more convenient to work with
offset representation.  The procedure {\sf offset-$>$index} converts a
location from offset to index representation.

The Scheme tokenizer ({\sf read-token}) returns a text buffer location
as an offset, so {\sf offset-$>$index} is frequently called to convert
such locations into indexes.

See the {\sc SWL} documentation for more specific information about
\cname{text}.
\end{class}



\begin{class}{scheme-text}{text}{Support}

\cname{scheme-text} is a direct subclass of \cname{text} that provides
support for display and editing of Scheme expressions.  Code
visualization aids such as parenthesis balancing, code indentation,
and lexical region marking (boxing) are provided here.

The expression collapser/outliner feature developed for the designer
project is actually implemented here.  Within \cname{scheme-text} the
feature is known as ``outline mode.''  The {\sf mouse-press}, {\sf
mouse-release}, and {\sf mouse-motion} methods are all specialized to
implement the collapser/outliner.
\end{class}




\begin{class}{edit-text}{scheme-text}{Editor}

\cname{edit-text} adds basic editor functionality to
\cname{scheme-text}, such as expression searching and cut/copy/paste.
\end{class}




\begin{class}{design-text}{edit-text}{Designer}

\cname{design-text} adds support for Locked and Unlocked designer modes,
and hooks for invoking the design parser in Locked Mode.

\begin{method}{add-version}{ tag}{}
Creates a new node in the version tree (see procedure {\sf
vtree-add-version} below) tagged with argument {\sl tag} (a string),
and with the entire current contents of the edit-buffer as the value
(a string).  The new node becomes a child of the currently selected
node (given by instance variable {\sf curr-version}) of the version
tree.
\end{method}
\end{class}






\begin{class}{voval}{oval}{Designer}


\cname{voval} adds support for tree nodes drawn in the Design History window. A
\cname{voval} instance also represents a single design version.


\begin{method}{mouse-press}{ x y modifiers}{}

Switches the designer from the current design (i.e., the previously
selected design version) to {\sf self} (i.e., the newly selected
design version).
\end{method}


\begin{method}{draw-marker}{}{}
Draws the square (the ``marker'') that denotes the currently selected
node in the version tree.
\end{method}
\end{class}




\begin{class}{vline}{line}{Designer}

\cname{voval} adds support for version tree arcs drawn in the Design
History window.
\end{class}



\begin{class}{vtool}{canvas}{Designer}

\cname{voval} adds support to the canvas widget, which is used as the
container for the Design History tree diagram.

\begin{method}{redraw}{}{}
Deletes all existing graphical objects (lines, rectangles, ovals) in
the Design History window, which denote the old version tree, and
redraws the entire version tree.
\end{method}
\end{class}






\begin{class}{markup}{}{Core}

\cname{markup} implements the basic functionality of Tk tags, which
enable portions of text contained in a \cname{text} instance to
displayed with emphasis (color or alternate font), and to be {\em
reactive} (e.g., responsive to mouse clicking).  A \cname{markup}
instance is associated with one or more \cname{text} instances by
invoking the \cname{markup} method {\sf apply-markup} with the text
widget instance as an argument.

\cname{markup} instances are used extensively in the designer in
Locked Mode to denote lexical variable references and bindings, and to
provide mouse functions to these subexpressions.
\end{class}



\begin{class}{collapsoid}{markup}{Support}

\cname{collapsoid} implements the program collapser/outliner feature,
which permits program subexpression forms to be collapsed to a point
(a single highlighted character), and later recovered (expanded) with
a mouse click.  Collapsed expressions may be nested inside another
collapsed region.

\begin{method}{collapse}{ twgt ix match cclass}{}

Collapses the Scheme expression in text widget {\sl twgt} between location
{\sl ix} and location {\sl match}.  The expression is replaced by a single
character (currently '\$'), highlighted in blue.  {\sl cclass} is unused.

\end{method}



\begin{method}{reapply}{ ix}{}

When a collapsoid is uncollapsed, reapply restores all nested
collapsoids that had been removed when the nesting collapsoid was
created.

\end{method}



\begin{method}{move-to}{ ix}{}

Relocates the collapsoid to index {\sl ix}.

\end{method}


\begin{method}{change-to-nested}{ ix}{}
	
Marks the collapsoid as nested within another collapsoid.

\end{method}


\begin{method}{expand}{}{}

Expands the collapsoid, restoring the original text and 
reapplying any nested collapsoids.

\end{method}


\begin{method}{expand-all}{}{}

Expands the collapsoid and all nested collapsoids.

\end{method}
\end{class}





\begin{class}{design-markup}{markup}{Designer}

\cname{design-markup} adds designer-specific functionality to markups,
such as meaning-preserving {\sf let} binding relocation, highlighting and
blinking of bindings and lexical variables, and substitution of
binding values for references.

In Locked Mode, every lexical variable declaration, {\sf let} binding, and
lexical variable reference is highlighted in color (see table
\ref{tab:highlight}).  In addition, moving the mouse cursor over a
{\sf let} binding will cause the entire binding to be highlighted.
All such functionality is implemented in terms of design markups.

A single design markup encapulates a single expression (program text)
in the user's design, the subexpressions of the expression if any, and
the current lexical environment, all of which were obtained during the
parsing of the design.  This information is combined with enough GUI
information (e.g., the row and column location of the expression in
the text buffer) to permit the highlighting and mouse capabilities to
be implemented.

For example, a single {\sf let} binding in a user's design will be
represented by (at least) two design markups: one to hold the entire
binding (with the bound variable and binding right-hand side as
subexpressions), and one to hold the bound variable itself (with a
null subexpression). If the right-hand side itself contains variable
references, additional design markups will be created for these as
well.


\begin{method}{move-to}{ dstidx}{}

Moves the expression bound to the current instance to screen location
{\sl dstidx} (a row, column pair).

\end{method}


\begin{method}{mouse-press}{ text x y mods}{}

Handles a mouse button click when the cursor is placed on a
\cname{design-markup} in the designer's text window.

The actual behavior of {\sf mouse-press} depends on the type of {\sf
design-markup} encountered, e.g., a lexical variable reference is
handled differently from a binding.

\end{method}


\begin{method}{mouse-enter}{ text x y mods}{}

Handles the highlighting (coloring) of certain \cname{design-markup}
types, such as those for lexical variable references and bindings,
when the mouse cursor moves over the \cname{design-markup}.

\end{method}


\begin{method}{mouse-leave}{ text x y mods}{}

Complementary to {\sf mouse-enter}, handles unhighlighting of certain
\cname{design-markup} types that are highlighted upon cursor entry.

\end{method}



\begin{method}{enable-blinking}{}{}

Causes the current \cname{design-markup} to start blinking in a
contrasting color.  Blinking is implemented using a separate {\sc SWL}
thread.

\end{method}


\begin{method}{disable-blinking}{}{}

Turns off blinking of the current \cname{design-markup}.

\end{method}
\end{class}



\section{Auxiliary Procedures}

While much of the procedural code in the designer is in the form of
methods defined within classes, some important functionality is
defined outside the class system, as Scheme global procedures.



\begin{proc}{make-popup}{ pop-ttls pop-acts}{}

{\sf make-popup} creates and returns a popup menu.  {\sl pop-ttls} is a Scheme
list of strings that form the popup menu items.  {\sl pop-acts} is a Scheme
list of thunks that perform the actions corresponding to the menu
items.

{\sf make-popup} is called by the mouse-press method of \cname{design-markup}, for
those \cname{design-markup} instances that have associated popup menus (currently only
lexical variable references).
\end{proc}


\begin{proc}{make-pretty}{ expr-str}{a pretty-printed string}

{\sf make-pretty} takes a Scheme expression as a string, {\sl
expr-str}, and applies the Chez Scheme pretty-printer to it, which
elides most of the expression's substructure.  The resultant string is
then suitable for display in the Design History window.

\end{proc}



\begin{proc}{new-design}{}{}

Executes the designer within a new {\sc SWL} window.  Multiple designers
may execute concurrently.

\end{proc}



\begin{proc}{occurs-free-except-for?}{ var expr except}{}

{\sf occurs-free-except-for?} determines whether variable {\sl var} (a
Scheme symbol) occurs free in expression {\sl expr}, with
subexpression {\sl except} excluded from the computation.  Used to
determine whether moving a binding is semantically correct.

\end{proc}



\begin{proc}{parse-it}{ text-wgt}{a boolean}

A key procedure in the designer, {\sf parse-it} performs a recursive
descent (top-down) parsing of the contents of the text widget buffer,
which is assumed to contain a valid design.  {\sl text-wgt} must be
the {\sc SWL} text widget instance that contains the user's design
text. {\sf parse-it} returns {\sf \#t} if parsing was successful, else
{\sf \#f}.  The LL(1) grammar for the parser is given in table
\ref{tab:grammar}.

The parser uses the conventional division of textual analysis into
separate lexical scanning and parsing: lexical scanning is done by the
procedure {\sf read-token} (described below), which converts
characters of design text into Scheme tokens. The output of {\sf
read-token} is then used as the input to {\sf parse-it}.  {\sf
read-token} also provides the row and column information on each token
scanned, which is essential for implementing the interactive features
of the designer.

As a side-effect, {\sf parse-it} assigns the parse tree to {\sf
text-wgt}. A parse tree is a set of \cname{design-markup} instances,
each of which encapsulates a large amount of design information:

\begin{itemize}

\item the starting and ending row and column of the expression in the
text buffer, and the parse token obtained from the lexical scanner
({\sf read-token}),

\item the current lexical environment, i.e., the current set of
lexical variables declared at any point during parsing,

\item the syntactic category (or grammar nonterminal) for the
expression (e.g., datum, binding),

\item a list of {\em holes,} i.e., the regions of buffer text that
belong to subexpressions of the current expression represented by
nested \cname{design-markup} instances, used to ensure that any nested
\cname{design-markup} instances are not shadowed by the current
\cname{design-markup}.

\item the parent \cname{design-markup}, if any, (else {\sf \#f})

\item the entire parse tree as a Scheme expression, (a symbol if the
parse tree is a terminal, else a Scheme list of parse sub-trees).

\item a symbol if the parse tree is a terminal, else a list of the 
\cname{design-markup} instances that represent the parse sub-trees.

\end{itemize}

As another side-effect, {\sf parse-it} invokes the {\sf apply-markup}
\cname{text} method for each \cname{design-markup} instance, causing
regions of text to take on highlighting (color) and to be become
sensitive to mouse input.

As additional side-effects, if {\sf parse-it} cannot parse its input,
it marks the offending input token in red and displays an error
message in the mini-buffer.

\end{proc}



\begin{proc}{read-token}{ ofs . ip}{\sl type, token, start, end}

{\sf read-token} reads the next Scheme lexical token from input port
{\sl ip}, or from {\sf current-input-port} if the optional argument
{\sl ip} is not supplied.  

{\sf read-token} returns four values: the type of the token in {\sl
type} (e.g., {\sf atomic} for a symbol), the token itself in {\sl
token} (e.g., the name of a symbol), and the starting and ending
character positions of the token in the input in {\sl start} and {\sl
end}, counting from the initial offset argument {\sl ofs}.

See the documentation file, {\sf read-token.doc}, for more
information.

\end{proc}



\begin{proc}{show-popup}{ pop x y}{}

{\sf show-popup} causes a popup menu {\sl pop}, created by {\sf
make-popup}, to be displayed at screen coordinates {\sl x} and {\sl
y}.  The coordinates are absolute, i.e., always from the upper left
corner of the display screen.

\end{proc}



\begin{proc}{vtree-add-version}{ v cv vtree}{a new vtree}
Accepts versions {\sl v} and {\sl cv}, and version tree {\sl vtree},
and returns the following:

\begin{itemize}
\item if {\sl v} is a version in {\sl vtree}, returns a new vtree with {\sl cv} as
     a child of {\sl v},

\item else, returns the original vtree.

\end{itemize}

A {\em version} is an instance of the {\sl version} structure, defined by:

\begin{scheme}
(define-structure (version tag item))
\end{scheme}

{\sl tag} is a Scheme string denoting the kind of transformation that
yielded the version, e.g., ``subst'').  {\sl item} is a Scheme string that
is the entire contents of the edit-buffer for that version.

A {\em vtree} is an instance of the {\sl vtree} structure, defined by:

\begin{scheme}
(define-structure (vtree version children))
\end{scheme}

A {\sl vtree} is an n-ary tree defined in the conventional way: an
n-ary tree is a single ``node,'' together with zero or more n-ary
trees as children.

{\sl version} is the attribute of a single node of a vtree.  {\sl
children} is a Scheme list of zero or more child vtrees.

\end{proc}






\section{Program Files and Pathnames}


Much of the designer implementation is in a single file, {\sf
design.ss}.  However, key portions of the designer are defined in
separate files, as follows.


\begin{description}


\item[../common/scrollframe.ss]

Provides the horizontal and vertical scrollbar capability for the text
widget.


\item[../common/fontmenu.ss]

Provides the font submenu for the menu-bar.


\item[../common/select-file.ss]

Provides a file-selection box capability, which is used for loading
and saving design files.


\item[../common/warning-dialog.ss]

Provides a dialog box capability, which is used for issuing warning
messages to the user.


\item[../common/auxio.ss]

Provides certain useful text-widget and file I/O utilities.


\item[../common/worrier.ss]

Provides the {\em worried face} icon, which is displayed whenever the text
buffer is modified but not saved to disk.


\item[../common/scheme-text.ss]

Provides basic Scheme language editing support for the text-widget, as
\cname{scheme-text}, a subclass of \cname{text}.


\item[../common/worrier.ss]

Provides the {\em worried face} icon, which is displayed whenever the text
buffer is modified but not saved to disk.


\item[../edit/edit-text.ss]

Provides the basic Scheme text-widget functionality for the text
editor application, as \cname{edit-text}, a subclass of
\cname{scheme-text}.


\item[read-token.ss]

Provides the position-tracking Scheme lexical scanner, {\sf
read-token}.


\item[read-token.doc]

Provides documentation for the position-tracking Scheme lexical
scanner, {\sf read-token}.


\item[popup.ss]

Provides a popup menu capability.


\item[dm.ss]

Provides the \cname{design-markup} class and its auxiliary procedures.

\item[vtool.ss]

Provides the functionality for the Design History window.

\end{description}





\section{An LL(1) Grammar for the Designer}

Table \ref{tab:grammar} below defines the LL(1) grammar used by the
designer.  The names of the nonterminals forms the set of syntactic
categories; the variable {\sf syncat} of each \cname{design-markup}
instance is bound to one of these symbols. The instance variable {\sf
exprdm} corresponds to the right-hand side of the grammar rule; it is
bound to a list structure containing \cname{design-markup} instances
for each non-terminal that appears in the right-hand side.


\begin{table}\centering
<
\begin{verbatim}
Expr     -> Form  | Literal | Varref | AbbrevQuote
Form     -> ( Lambda | ( Define | ( Let... | ( If | ( QuoteForm | 
            ( App | ()
  Lambda   -> lambda ( Formals* ) Expr ) ;; no sequences
  Define   -> define Var Expr )
  Let      -> let Bindings Expr ) ;; no sequences
  LetStar  -> let* Bindings Expr ) ;; no sequences
  Letrec   -> letrec Bindings Expr ) ;; no sequences
  If       -> if Expr1 Expr2 Expr3 ) | ( if Expr1 Expr2 )
  QuoteForm-> quote Datum )
  App      -> Expr+ )
Expr+    -> Expr | Expr Expr+    
AbbrevQuote-> ' Datum
Bindings -> ( Bindings* )
Bindings*-> <epsilon> | Binding Bindings*
Binding  -> ( Var Expr )
Formals* -> <epsilon> | Var Formals*
Body*    -> <epsilon> | Expr Body* ;; currently unused
Datum    -> Literal | Sym | ( Datum* ) | ( Datum Datum* . Datum )
Datum*   -> <epsilon> | Datum Datum*
Literal  -> num | char 
\end{verbatim}

\caption{An LL(1) Grammar for the Designer}\label{tab:grammar}
\end{table}




\section{Acknowledgements}

Oscar Waddell, chief architect of {\sc swl}, collaborated closely the
author.  R. Kent Dybvig supervised the overall {\sc swl} project and
is the author of Chez Scheme.  Steve Johnson provided the research
direction for the designer.




\begin{thebibliography}{99}

\bibitem{TSPL} R. Kent Dybvig. {\em The Scheme Programming Language},
Second Edition, Prentice-Hall, Inc., Upper Saddle River, New Jersey,
1996.

\bibitem{CSSM} R. Kent Dybvig.  {\em Chez Scheme System Manual,
Revision 2.4}.  Cadence Research Systems, Bloomington, Indiana. 1994.

\bibitem{TCLTK} Ousterhout, J. {\em Tcl/Tk}, Sun Microsystems
Laboratories.

\bibitem{SWL} Waddell, O. {\em The SWL Manual}, Indiana University, 1997.



\end{thebibliography}







\end{document}

